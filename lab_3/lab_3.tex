%%%%%%%%%%%%%%%%%%%%%%%%%%%%%%%%%%%%%%%%%
% University/School Laboratory Report
% LaTeX Template
% Version 3.1 (25/3/14)
%
% This template has been downloaded from:
% http://www.LaTeXTemplates.com
%
% Original author:
% Linux and Unix Users Group at Virginia Tech Wiki 
% (https://vtluug.org/wiki/Example_LaTeX_chem_lab_report)
%
% License:
% CC BY-NC-SA 3.0 (http://creativecommons.org/licenses/by-nc-sa/3.0/)
%
%%%%%%%%%%%%%%%%%%%%%%%%%%%%%%%%%%%%%%%%%

%----------------------------------------------------------------------------------------
%	PACKAGES AND DOCUMENT CONFIGURATIONS
%----------------------------------------------------------------------------------------

\documentclass{article}

\usepackage[version=3]{mhchem} % Package for chemical equation typesetting
\usepackage{siunitx} % Provides the \SI{}{} and \si{} command for typesetting SI units
\usepackage{graphicx} % Required for the inclusion of images
\usepackage{natbib} % Required to change bibliography style to APA
\usepackage{amsmath} % Required for some math elements 
\usepackage{enumerate} % Required for the enumerate function
\usepackage[siunitx]{circuitikz} % Required for the drawing of circuit diagrams
\usepackage{caption}
\usepackage{graphicx}
\usepackage{subcaption}
\usepackage{xfrac}
\usepackage{float}
\usepackage{enumitem}
\usepackage{chemgreek}
\usepackage{pgfplots}
\usepackage{booktabs}
\usepackage[]{mcode}
\usepackage{epstopdf}
\usepackage{tikz}
\usetikzlibrary{arrows}
\usepackage{booktabs}
\usepackage{capt-of}
\usepackage{ mathrsfs }

\setlength\parindent{0pt} % Removes all indentation from paragraphs

\renewcommand{\labelenumi}{\alph{enumi}.} % Make numbering in the enumerate environment by letter rather than number (e.g. section 6)

%\usepackage{times} % Uncomment to use the Times New Roman font

\graphicspath{{./fig/}}

%----------------------------------------------------------------------------------------
%	DOCUMENT INFORMATION
%----------------------------------------------------------------------------------------

\title{Communication Systems \\ BCH and Turbo BCH Codes \\ ENG473} % Title

\author{Shane \textsc{Reynolds}} % Author name

\date{October 12, 2016} % Date for the report

\begin{document}

\maketitle % Insert the title, author and date

\begin{center}
\begin{tabular}{l r}
Date Performed: & October 12, 2016 \\ % Date the experiment was performed
Instructor: & Dr Sina Vafi % Instructor/supervisor
\end{tabular}
\end{center}

% If you wish to include an abstract, uncomment the lines below
% \begin{abstract}
% Abstract text
% \end{abstract}

%----------------------------------------------------------------------------------------
%	SECTION 1
%----------------------------------------------------------------------------------------

\section{Objective}

Gain understanding of the functional operation of binary BCH and binary BCH turbo codes using a simulated code generation to obtain the minimum Hamming distance and Hamming weight as performance metrics.

%----------------------------------------------------------------------------------------
%	SECTION 2
%----------------------------------------------------------------------------------------

\section{Part 1: Simulating BCH(7,4) Codes}

BCH codes form a class of cyclic error correcting codes, and as such the code words take on the following form:
\begin{align*}
	X = [M \ | \ C]
\end{align*}

The vector $M$ is the information being sent, in this case a binary digit vector of length 4. The total number of permutations that the 4-bit vector can take on is 16. The vector $C$, is a parity check vector, which is determined by the BCH encoder. In this instance, the generator polynomial used for encoding is given by:
\begin{align*}
	G(p) = p^3 + p + 1
\end{align*}

The BCH encoder can be implemented using a shift register approach, which can be seen in Figure 1.


\tikzstyle{int}=[draw, fill=blue!20, minimum height=4em, text width=5em, text centered]
\tikzstyle{sum} = [draw, fill=white, circle, node distance=1cm]
\tikzstyle{init} = [pin edge={to-,thin,black}]
\begin{figure}[H]
	\hspace{-2.5cm}
	\begin{tikzpicture}[node distance=2.5cm,auto,>=latex']
	\node [sum] (b) {$+$};
	\node (a) [left of=b,node distance=2cm, coordinate] {input};
	\node [int] (c) [right of=b] {$r_0$};
	\node [sum] (d) [right of=c, node distance=2.5cm] {$+$};
	\node [int] (e) [right of=d] {$r_1$};
	\node [int] (f) [right of=e, node distance=3.5cm] {$r_2$};
	\node [coordinate] (end) [right of=f, node distance=3.5cm]{};
	\node [] (x) [above of=b] {};
	\node [] (y) [above of=d, inner sep=0pt, outer sep=-0.5pt] {};
	
	\path[->] (a) edge node [label=below:Input] {} (b);
	\path[->] (b) edge node {} (c);
	\path[->] (c) edge node {} (d);
	\path[->] (d) edge node {} (e);
	\path[->] (e) edge node {} (f);
	\draw [->] (f) -- node [name=n, inner sep=0pt, outer sep=-0.5pt, label=below:Output] {}(end);
	%\path[->] (f) edge node {} (end);
	\draw [-] (n) |- (x);
	\draw [->] (x) -| (b);
	\path [->] (y) edge node {} (d);
	\end{tikzpicture}
	\caption{Shift register implementation of the BCH(7,4) encoder with generator polynomial $G(p) = p^3 + p + 1$}
\end{figure}

The shift register structure was used to determine the BCH code for two messages: 1011 and 0111. The results of the analysis can be seen in Tables 1 and 2, respectively.

\begin{table}[H]
	\centering
	\caption{Contents of shift register for an input message 1011.}
	\begin{tabular}{cccccccc}
		\toprule
		\textbf{Input} & \multicolumn{3}{c}{\textbf{Prev. State}} & \multicolumn{3}{c}{\textbf{Next State}} & \textbf{Output}\\
		$z(p)$ & $r_0$ & $r_1$ & $r_2$ & $r_0$ & $r_1$ & $r_2$ & \\
		\midrule
		1	&	0	&	0	&	0	&	1	&	0	&	0	&	0\\
		0	&	1	&	0	&	0	&	0	&	1	&	0	&	0\\
		1	&	0	&	1	&	0	&	1	&	0	&	1	&	0\\
		1	&	1	&	0	&	1	&	0	&	0	&	0	&	1\\
		0	&	0	&	0	&	0	&	0	&	0	&	0	&	0\\
		0	&	0	&	0	&	0	&	0	&	0	&	0	&	0\\
		0	&	0	&	0	&	0	&	0	&	0	&	0	&	0\\
		\bottomrule
	\end{tabular}
\end{table}

\begin{table}[H]
	\centering
	\caption{Contents of shift register for an input message 0111.}
	\begin{tabular}{cccccccc}
		\toprule
		\textbf{Input} & \multicolumn{3}{c}{\textbf{Prev. State}} & \multicolumn{3}{c}{\textbf{Next State}} & \textbf{Output}\\
		$z(p)$ & $r_0$ & $r_1$ & $r_2$ & $r_0$ & $r_1$ & $r_2$ & \\
		\midrule
		0	&	0	&	0	&	0	&	0	&	0	&	0	&	0\\
		1	&	0	&	0	&	0	&	1	&	0	&	0	&	0\\
		1	&	1	&	0	&	0	&	1	&	1	&	0	&	0\\
		1	&	1	&	1	&	0	&	1	&	1	&	1	&	0\\
		0	&	1	&	1	&	1	&	1	&	0	&	1	&	1\\
		0	&	1	&	0	&	1	&	1	&	0	&	0	&	1\\
		0	&	1	&	0	&	0	&	0	&	1	&	0	&	0\\
		\bottomrule
	\end{tabular}
\end{table}

The input message 1011 has the BCH codeword 1011000, and the input message 0111, has the codeword 0111010. The BCH encoding scheme was implemented in MATLAB - the code can be found in Appendix A. A BCH codeword was generated for each of the sixteen 4-bit permutations using this implementation. Further, the Hamming weight was found for each of the codewords. The results can be seen in Table 3. The minimum Hamming distance, $d_{min}$, was calculated using a MATLAB function which can be seen in Appendix A. The minimum Hamming distance was found to be:
\begin{center}
	\fbox{$d_{min} = 3$}
\end{center}

\begin{figure}[H]
	\hspace{-0.5cm}
	\begin{minipage}{7cm}
		\centering
		\captionof{table}{Full set of codewords for the\\ BCH(7,4) code.}
		\begin{tabular}{ccc}
			\toprule
			$M$ & $C$ & Weight\\
			\midrule
			0000	&	0000000 	&	0 	\\
			0001	&	0001011		&	3	\\
			0010	&	0010110		&	3	\\
			0011	&	0011101		&	4	\\
			0100	&	0100111		&	4	\\
			0101	&	0101100		&	3	\\
			0110	&	0110001		&	3	\\
			0111	&	0111010		&	4	\\
			1000	&	1000101		&	3	\\
			1001	&	1001110		&	4	\\
			1010	&	1010011		&	4	\\
			1011	&	1011000		&	3	\\
			1100	&	1100010		&	3	\\
			1101	&	1101001		&	4	\\
			1110	&	1110100		&	4	\\
			1111	&	1111111		&	7	\\
			\bottomrule
		\end{tabular}
	\end{minipage}
	\begin{minipage}{7cm}
		\centering
		\captionof{table}{Full set of codewords for the\\ turbo BCH(7,4) code.}
		\begin{tabular}{ccc}
			\toprule
			$M$ & $C$ & Weight\\
			\midrule
			0000	&	0000000000 		&	0 	\\
			0001	&	0001001111		&	5	\\
			0010	&	0010111101		&	6	\\
			0011	&	0011110010		&	5	\\
			0100	&	0100111110		&	6	\\
			0101	&	0101110001		&	5	\\
			0110	&	0110000011		&	4	\\
			0111	&	0111001100		&	5	\\
			1000	&	1000110011		&	5	\\
			1001	&	1001111100		&	6	\\
			1010	&	1010001110		&	5	\\
			1011	&	1011000001		&	4	\\
			1100	&	1100001101		&	5	\\
			1101	&	1101000010		&	4	\\
			1110	&	1110110000		&	5	\\
			1111	&	1111111111		&	10	\\
			\bottomrule
		\end{tabular}
	\end{minipage}
\end{figure}
\newpage
%----------------------------------------------------------------------------------------
%	SECTION 3
%----------------------------------------------------------------------------------------

\section{Part 2: Simulating BCH(7,4) Turbo Codes}
Turbo BCH uses two identical BCH encoders, however, the information message is parsed through a row-column interleaver prior to being encoded by the BCH encoder. The codeword that is produced is again of the form:
\begin{align*}
	X = [M \ | \ C]
\end{align*}

The vector $M$ is the 4-bit information message to be sent, and the vector $C$ is the parity vector. The generator polynomial used for this simulation was:
\begin{align*}
	G(p) = p^ 3 + p +1
\end{align*}

Turbo BCH differs from the standard BCH in that the vector $C$ is made up of components from the parity vector from the first BCH process, and from the second BCH process. Figure 2 shows a block diagram demonstrating the make up of the codewords.

\tikzstyle{int}=[draw, fill=blue!20, minimum height=4em, text width=5em, text centered]
\tikzstyle{sum} = [draw, fill=white, circle, node distance=1cm]
\tikzstyle{init} = [pin edge={to-,thin,black}]
\begin{figure}[H]
	\hspace{-0.5cm}
	\begin{tikzpicture}[node distance=2.5cm,auto,>=latex']
	\node [int] (b) {BCH Code 1};
	\node (a) [left of=b,node distance=6cm, coordinate] {input};
	\node [int] (c) [below of=b] {BCH Code 2};
	\node [coordinate] (d) [right of=b, node distance=5.5cm]{};
	\node [coordinate] (e) [right of=c, node distance=5.5cm]{};
	\node [coordinate] (end) [below of=d, node distance=1.5cm, label=$m_0 m_1 m_2 m_3 c_0 c'_0 c_1 c'_1 c_2 c'_2$]{};
	%\node [coordinate] (end1) [right of=end, label=$m_0 m_1 m_2 m_3 c_0 c'_0 c_1 c'_1 c_2 c'_2$]{};
	
	\draw [->] (a) -- node [name=n, inner sep=0pt, outer sep=-0.5pt] {}(b);
	\node [int] (m) [below of=n]{row-column interleaver};
	\node (z) [above of=n, node distance=0.25cm] {$m_0 m_1 m_2 m_3$};
	\path [->] (n) edge node {} (m);
	\path [->] (m) edge node {} (c);
	\path [->] (b) edge node {$m_0 m_1 m_2 m_3 c_0 c_1 c_2$} (d);
	\path [->] (c) edge node {$c'_0 c'_1 c'_2$} (e);
	\end{tikzpicture}
	\caption{Block diagram demonstrating construction of the turbo BCH using a row-column interleaver.}
\end{figure}

The turbo BCH code was simulated in MATLAB - the implementation can be seen in Appendix B. A turbo BCH code was generated for each of the sixteen 4-bit code words using this implementation. Further, the Hamming weights were calculated for each of the codewords. The results can be seen in Table 4 on the previous page. The minimum Hamming weight that was found, apart from 0, was:
\begin{center}
	\fbox{Min Hamming weight = 4}
\end{center}

\vspace{0.5cm}

The minimum Hamming distance was found to be: 
\begin{center}
	\fbox{$d_{min} = 4$}
\end{center}

%----------------------------------------------------------------------------------------
%	SECTION 4
%----------------------------------------------------------------------------------------

\section{Part 3: Simulating BCH(15,11) and Turbo BCH(15,11)}
BCH(15,11) code was generated again using a MATLAB simulation. The implementation can be seen in Appendix C. This time, however, the bit stream was 22-bit in length and hence had to be split into 2 11-bit words. The generator polynomial used for the implementation was given as:
\begin{align*}
	G(p) = p^4 + p + 1
\end{align*}

Each word was converted to BCH(15,11), and then they were appended serially to form a 30-bit codeword. The minimum Hamming distance was calculated as follows:
\begin{center}
	\fbox{$d_{min} = 3$}
\end{center}

Turbo BCH(15,11) code was also generated using a MATLAB simulation. The implementation can be seen in Appendix D. The interleaving component of the turbo BCH had to be padded out with zeros in order to make the code work. These were stripped away afterwards to arrive at the answer. The minimum Hamming distance was calculated as follows:
\begin{center}
	\fbox{$d_{min} = 5$}
\end{center} 
%----------------------------------------------------------------------------------------
%	SECTION 5
%----------------------------------------------------------------------------------------

\section{Conclusion}
The turbo BCH code outperforms the BCH code in terms of minimum Hamming distance. This means that the turbo BCH code is better equipped to deal with the correction of error in the transmitted codewords. These results are further confirmed in the BCH(15,11) and turbo BCH(15,11) code simulations shown in part 3 of the report. We note that given the additional parity bits, the minimum Hamming distance did not improve. Very simply the code's error correcting capabilities are bounded by the minimum Hamming distance metric, such that $d_{min} \geq 2t + 1$. Hence, the BCH code can only correct 1 error, however, the turbo BCH is closer to correcting 2 errors, which explains the similar performance seen in both code sets.\\

Code performance, is in part, due to the greater number of parity bits that the code contains, but also due to the interleaving mechanism. According to Wikipedia, interleaving distributes source symbols across several codewords, and creates a more uniform distribution of errors. This is a desirable property in digital transmission as errors tend to be occur in clusters which are called error bursts. 

\newpage
%----------------------------------------------------------------------------------------
%	APPENDIX A
%----------------------------------------------------------------------------------------

\section{Appendix A}
The following script generates the full set of BCH(7,4) codewords, finds the Hamming weights, and calculates the minimum Hamming distance with the use of a function:
\begin{lstlisting}
	%%%%%%%%%%%%%%%%%%%%%%%%%%%%%%%%%%%%%%%%%%%%%%%
	% BCH (7,4) Code Simulation and Code Generation
	%%%%%%%%%%%%%%%%%%%%%%%%%%%%%%%%%%%%%%%%%%%%%%%
	
	% Clear the workspace and any variables
	clear, clc;
	
	% Set the BCH code parameters
	n = 7;
	k = 4;
	
	% Set up BCH encoder and decoder
	enc = comm.BCHEncoder(n,k,'x3+x+1');
	%dec = comm.BCHDecoder(n,k,'x3+x+1'); 
	
	
	for dec_num = 0:(2^k - 1)
	
		% Convert the decimal number to the 4-bit word in matrix form
		x = dec2bin(dec_num,k);
		for i = 1:length(x)
			X(i) = str2num(x(i));
		end
	
		% Encode with BCH
		codeword = step(enc,X');
		codeword_mat(:,dec_num+1) = codeword;
		% Determine the weight of the BCH code
		weight = sum(codeword);
	
		%fprintf('*************************************\n')
		fprintf(['The message: ',x]); fprintf('\n');
		codeword'
		weight
		fprintf('*************************************\n')
	
	end
	
	dmin = dmincalc(codeword_mat);
	fprintf('Min distance: %d\n', dmin);
\end{lstlisting}
\newpage

The following function calculates the minimum Hamming distance:
\begin{lstlisting}
	function dmin = dmincalc(M)
	%%%%%%%%%%%%%%%%%%%%%%%%%%%%%%%%%%%%
	% Min Hamming Distance Calculator
	% Input: complete codeword matrix
	%%%%%%%%%%%%%%%%%%%%%%%%%%%%%%%%%%%%
		[r,c] = size(M);
	
		% Prespecify best possible Hamming distance
		dmin = 7;
	
		for i = 1:(c-1)
			for j = (i+1):c
				%fprintf('%d %d\n',i,j)
				dist = sum(mod(M(:,i) + M(:,j),2));
				% Store progressively worse Hamming distance
				if dist < dmin
					dmin = dist;
				end
			end
		end
	end
\end{lstlisting}

\newpage
%----------------------------------------------------------------------------------------
%	APPENDIX B
%----------------------------------------------------------------------------------------

\section{Appendix B}
The following script generates the full set of turbo BCH(7,4) codewords, finds the Hamming weights, and calculates the minimum Hamming distance with the use of a function (listed in Appendix A):
\begin{lstlisting}
	%%%%%%%%%%%%%%%%%%%%%%%%%%%%%%%%%%%%%%%%%%%%%%%%%%%%%%%
	% Turbo BCH(7,4) code simulation and codeword generation
	%%%%%%%%%%%%%%%%%%%%%%%%%%%%%%%%%%%%%%%%%%%%%%%%%%%%%%%
	
	% Clear the workspace and any stored variables
	clear, clc;
	n = 7;
	k = 4;
	
	% Set up BCH encoder and decoder
	enc = comm.BCHEncoder(n,k,'x3+x+1');
	%dec = comm.BCHDecoder(n,k,'x3+x+1');
	
	for dec_num = 0:(2^k - 1)
		% Convert the decimal number to the 4-bit word in matrix form
		x = dec2bin(dec_num,k);
		for i = 1:length(x)
			X(i) = str2num(x(i));
		end
	
		% Create interleaved input
		X_intr = matintrlv(X,2,2);
		% Encode first stream with BCH
		codeword_1 = step(enc,X');
		% Encode second (interleaved) stream with BCH
		codeword_2 = step(enc,X_intr');
		% Specify empty partiy code storage
		par = [];
		% Create turbo BCH partiy codeword
		for i = (k+1):length(codeword_1)
			par = [par;codeword_1(i);codeword_2(i)];
		end
		% Create the actual codeword
		codeword = [codeword_1(1:k);par];
		codeword_mat(:,dec_num+1) = codeword;
		% Determine the weight of the BCH code
		weight = sum(codeword);
	
		%fprintf('*************************************\n')
		fprintf(['The message: ',x]); fprintf('\n');
		codeword'
		weight
		fprintf('*************************************\n')
	
	end
	
	% Find the min Hamming distance
	dmin = dmincalc(codeword_mat);
	fprintf('Min distance: %d\n', dmin);
\end{lstlisting}
%----------------------------------------------------------------------------------------
%	APPENDIX C
%----------------------------------------------------------------------------------------

\section{Appendix C}
\begin{lstlisting}
	%%%%%%%%%%%%%%%%%%%%%%%%%%%%%%%%%%%%%%%%%%%%%%%%%%%%%%%%%
	% BCH(15,11) Simulation and Codeword Generation
	%%%%%%%%%%%%%%%%%%%%%%%%%%%%%%%%%%%%%%%%%%%%%%%%%%%%%%%%%
	
	% Clear the workspace and any stored variables
	clear; clc;
	% Set the the BCH(15,11) parameters
	n = 15;
	k = 11;
	L = 22;
	
	% Set up BCH encoder and decoder
	enc = comm.BCHEncoder(n,k,'x4+x+1');
	%dec = comm.BCHDecoder(n,k,'x3+x+1'); 
	
	
	for dec_num = 0:(2^L - 1)
		% Convert the decimal number to the 22-bit word in matrix form
		x = dec2bin(dec_num,L);
		for i = 1:length(x)
			X(i) = str2num(x(i));
		end
		% Encode with BCH first 11
		codeword_1 = step(enc,X(1:11)');
		% Encode with BCH second 11
		codeword_2 = step(enc,X(12:22)');
		% Actual codeword
		codeword = [codeword_1;codeword_2];
		% Store codeword in matrix
		codeword_mat(:,dec_num+1) = codeword;
		% Determine the weight of the BCH code
		weight = sum(codeword);
	
		%fprintf('*************************************\n')
		%fprintf(['The message: ',x]); fprintf('\n');
		%codeword'
		%weight
		%fprintf('*************************************\n')
		if mod(dec_num,100000) == 0
			fprintf('*')
		end
	end
	
	% Find the minimum Hamming Distance
	dmin = dmincalc2(codeword_mat);
	fprintf('Min distance: %d\n', dmin);
\end{lstlisting}

\newpage


%----------------------------------------------------------------------------------------
%	APPENDIX D
%----------------------------------------------------------------------------------------

\section{Appendix D}

\begin{lstlisting}
	%%%%%%%%%%%%%%%%%%%%%%%%%%%%%%%%%%%%%%%%%%%%%%%%%%%%%%%%%%%%
	% BCH Turbo Code(15,11) Simulation & Code Generation
	%%%%%%%%%%%%%%%%%%%%%%%%%%%%%%%%%%%%%%%%%%%%%%%%%%%%%%%%%%%%
	
	% Clear the workspace and variables that are stored
	clear, clc;
	
	% Set the parameters for the Turbo BCH(15,11)
	n = 15;
	k = 11;
	L = 22;
	% Set up BCH encoder and decoder
	enc = comm.BCHEncoder(n,k,'x4+x+1');
	%dec = comm.BCHDecoder(n,k,'x3+x+1');
	
	for dec_num = 0:(2^L - 1)
		% Convert the decimal number to the 22-bit word in matrix form
		x = dec2bin(dec_num,L+2);
		for i = 1:L
			X(i) = str2num(x(i));
		end
		% Break up input into 2 11-bit words
		codepart_1 = X(1:11);
		codepart_2 = X(12:22);
		% Pad with extra zero at end for interleving
		codepart_1_int = [codepart_1 0];
		codepart_2_int = [codepart_2 0];
		% Create ro-column interleaved input
		X_intr_1 = matintrlv(codepart_1_int,3,4);
		X_intr_2 = matintrlv(codepart_2_int,3,4);
		% Strip off the extra 0
		X_intr_1 = X_intr_1(1:end-1);
		X_intr_2 = X_intr_2(1:end-1);
		% Encode with BCH1 first 11 and second 11 
		codeword_cp_1 = step(enc,codepart_1'); % First 11
		codeword_cp_2 = step(enc,codepart_2'); % Second 11
		% Encode with BCH2 using first interleaved and second interleaved 
		codeword_int_cp_1 = step(enc,X_intr_1');
		codeword_int_cp_2 = step(enc,X_intr_2');
		% Prespecify storage for partiy
		par_cp_1 = [];
		par_cp_2 = [];
		% Put together actual codewords
		for i = (k+1):length(codeword_cp_1)
			par_cp_1 = [par_cp_1;codeword_cp_1(i);codeword_int_cp_1(i)];
			par_cp_2 = [par_cp_2;codeword_cp_2(i);codeword_int_cp_2(i)];
		end
		% Piece codewords together
		codeword_1 = [codeword_cp_1(1:k);par_cp_1];
		codeword_2 = [codeword_cp_2(1:k);par_cp_2];
		% Actual codeword
		codeword = [codeword_1;codeword_2];
		% Store codeword in matrix
		codeword_mat(:,dec_num+1) = codeword;
		% Determine the weight of the BCH code
		weight = sum(codeword);
	
		%fprintf('*************************************\n')
		%fprintf(['The message: ',x]); fprintf('\n');
		%codeword'
		%weight
		%fprintf('*************************************\n')
		if mod(dec_num,100000) == 0
			fprintf('*')
		end
	end
	
	% Calculate the minimum Hamming Distance
	dmin = dmincalc2(codeword_mat);
	fprintf('Min distance: %d\n', dmin);
\end{lstlisting}


\end{document}